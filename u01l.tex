\documentclass[paper=a4, english, ngerman]{scrartcl}

\usepackage[a4paper,left=2cm,right=2cm,top=2.5cm,bottom=3cm]{geometry}
\usepackage[ngerman]{babel}
\usepackage{tabularx}
\usepackage[utf8]{inputenc}
\usepackage{multirow}
\usepackage{listings}
\usepackage{amsmath}
\usepackage{mathtools}
\usepackage{amssymb}
\usepackage{dsfont}
\usepackage{wasysym}
\usepackage{enumitem}
\usepackage{stmaryrd}
\usepackage{graphicx}
%\usepackage{hyperref}
\usepackage{url}

\parindent 0pt
\lstset{basicstyle={\linespread{0.3}\ttfamily\scriptsize}, tabsize=2, basewidth=1.3mm, language=java,
	literate=%
    {Ö}{{\"O}}1
    {Ä}{{\"A}}1
    {Ü}{{\"U}}1
    {ß}{{\ss}}1
    {ü}{{\"u}}1
    {ä}{{\"a}}1
    {ö}{{\"o}}1
    {~}{{\textasciitilde}}1
}

\begin{document}
	\begin{tabular}{p{8.5cm}ll}
		\multirow{2}{*}{\huge{ALP4 - Übung 1} } & Bearbeiter: &  Dor Cohen, Lucas Antelo Blanco \\
		                                 & Tutorium:   & Alexander K., Mo. 16 Uhr         \\ \hline
	\end{tabular}
	
\stepcounter{section}
\section{Aufgabe 1 Logik und diskrete Mathematik}

	\begin{enumerate}[label=\alph*)]
		\item Boolscher Ausdruck ist die explizite Beschreibung einer boolschen Funktion mit boolschen Operatoren.
		
		
		\item \neg (\forall i\in N : (i > 5) \Rightarrow (\exists j \in N : j+i > 100) ) \Longleftrightarrow
		\exists i \in N : (i>5) \Rightarrow (\forall j \in N : j+i <= 100)
		
		
	\end{enumerate}

\section{Zu viel Milch}

\begin{enumerate}[label=\alph*)]
	\item \begin{itemize}
		\item Erstes Verfahren: Eine Notize auf dem Kühlschrank zu hinterlassen.
		
		Korrektheit: mit diesem Verfahren weisst man immer, wenn die andere Person zum Einkaufen ging.
		
		\item Zweites Verfahren: Genau einen Pfad zum Laden vereinbaren.
		
		Korrektheit: mit diesem Verfahren, trifft man sich mit der anderen Person, falls sie beide zum Einkaufen gingen.
		
		
	
	

		

\end{document}
